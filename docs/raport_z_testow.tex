\documentclass[a4paper,11pt]{article}
\usepackage[utf8]{inputenc}
\usepackage[T1]{fontenc}
\usepackage[polish]{babel}
\usepackage{geometry}
\usepackage{longtable}
\usepackage{graphicx}
\usepackage{float}

\geometry{
 a4paper,
 total={170mm,257mm},
 left=25mm,
 top=25mm,
}

\title{\textbf{Raport końcowy z testów dla modułu Video-Sent API}}
\author{Kacper Przybylski, Kacper Witek, Kacper Stasiak,\\Julia Ruszer, Dawid Frontczak, Jakub Cendalski}
\date{\today}

\begin{document}

\maketitle

\section{Wstęp}
Niniejszy dokument przedstawia raport końcowy z testów dla modułu backendowego Video-Sent API, będącego częścią systemu analizy sentymentu wideo. Dokument jest wynikiem prac przeprowadzonych w ramach etapu weryfikacji jakości oprogramowania.
W kolejnych rozdziałach opisano wyniki prac związanych z realizacją zadań testowych oraz wnioski wynikające z ich przeprowadzenia.

\section{Podsumowanie procedury testowej}
W procesie testowym, zgodnie z Planem Testów, zostały przeprowadzone testy modułowe (jednostkowe), testy integracyjne API, testy wydajnościowe oraz testy bezpieczeństwa. Przetestowane zostały następujące elementy:
\begin{itemize}
    	\item Endpointy API (inicjowanie, status, wynik).
    	\item Serwisy backendowe (Downloader, Transcription, NLP).
    	\item Mechanizmy bazy danych (CRUD).
\end{itemize}
Testy zostały w całości zautomatyzowane przy użyciu frameworka \texttt{pytest} oraz narzędzi \texttt{locust} i \texttt{bandit}.

W ramach procedury testowej nie wykryto krytycznych defektów blokujących wydanie. Wykryte ryzyka (np. wydajność zewnętrznych API) zostały zmitigowane poprzez zastosowanie mechanizmów asynchronicznych.

\section{Odchylenia od planu testów}
Brak znaczących odchyleń. Testy wydajnościowe przeprowadzono na zamockowanych usługach zewnętrznych w celu uniknięcia kosztów operacyjnych, co było zgodne z ustaleniami.

\section{Ocena kompletności testów}
\begin{itemize}
    	\item Liczba pokrytych przypadków użycia: 100\%
    	\item Całkowita liczba zaimplementowanych testów automatycznych: 17
    	\item Całkowita liczba wykonanych testów: 17
    	\item Liczba testów zakończonych sukcesem: 17
    	\item Liczba testów zakończonych porażką: 0
\end{itemize}

\section{Raport z aktualnego stanu testów}

\subsection{Testy integracyjne API}
\begin{longtable}{|p{5.5cm}|p{7.5cm}|p{2cm}|}
\hline
\textbf{Nazwa testu} & \textbf{Opis testu} & \textbf{Status} \\
\hline
\endfirsthead
\hline
\textbf{Nazwa testu} & \textbf{Opis testu} & \textbf{Status} \\
\hline
\endhead

test\_initiate\_analysis & Weryfikacja poprawnego utworzenia zadania analizy dla nowego URL. & Zaliczony \\
\hline
test\_initiate\_analysis\_duplicate & Sprawdzenie obsługi duplikatów (zwrócenie istniejącego wyniku). & Zaliczony \\
\hline
test\_initiate\_analysis\_conflict & Sprawdzenie obsługi konfliktów (podpięcie pod trwające zadanie). & Zaliczony \\
\hline
test\_get\_status & Weryfikacja pobierania statusu zadania. & Zaliczony \\
\hline
test\_get\_status\_not\_found & Obsługa błędu 404 dla nieistniejącego zadania. & Zaliczony \\
\hline
test\_get\_result & Weryfikacja pobierania kompletnych wyników analizy. & Zaliczony \\
\hline
test\_get\_result\_not\_found & Obsługa błędu 404 dla nieistniejącego wyniku. & Zaliczony \\
\hline
\end{longtable}

\subsection{Testy jednostkowe (Unit Tests)}
\begin{longtable}{|p{5.5cm}|p{7.5cm}|p{2cm}|}
\hline
\textbf{Nazwa testu} & \textbf{Opis testu} & \textbf{Status} \\
\hline
\endfirsthead
\hline
\textbf{Nazwa testu} & \textbf{Opis testu} & \textbf{Status} \\
\hline
\endhead

test\_download\_audio\_success & Symulacja poprawnego pobrania pliku przez serwis Downloader. & Zaliczony \\
\hline
test\_download\_audio\_failure & Weryfikacja obsługi błędów biblioteki yt-dlp. & Zaliczony \\
\hline
test\_transcribe\_success & Symulacja poprawnej transkrypcji przez serwis OpenAI. & Zaliczony \\
\hline
test\_transcribe\_file\_not\_found & Walidacja istnienia pliku przed wysłaniem do API. & Zaliczony \\
\hline
test\_transcribe\_api\_error & Walidacja poprawności serwisu OpenAI. & Zaliczony \\
\hline
test\_analyze\_text\_logic & Weryfikacja logiki mapowania sentymentu na aspekty (NLP). & Zaliczony \\
\hline
test\_analyze\_text\_averaging & Weryfikacja uśredniania sentymentu dla wielokrotnych wzmianek o aspekcie (NLP). & Zaliczony \\
\hline
\end{longtable}

\subsection{Testy integracyjne NLP}
\begin{longtable}{|p{6cm}|p{7cm}|p{2cm}|}
\hline
\textbf{Nazwa testu} & \textbf{Opis testu} & \textbf{Status} \\
\hline
\endfirsthead
\hline
\textbf{Nazwa testu} & \textbf{Opis testu} & \textbf{Status} \\
\hline
\endhead
test\_nlp\_analysis\_iphone\_review & Weryfikacja analizy sentymentu dla przykładowej recenzji (iPhone). & Zaliczony \\
\hline
test\_nlp\_analysis\_negative\_review & Weryfikacja analizy sentymentu dla ogólnie negatywnego tekstu. & Zaliczony \\
\hline
test\_nlp\_analysis\_positive\_review & Weryfikacja analizy sentymentu dla ogólnie pozytywnego tekstu. & Zaliczony \\
\hline
\end{longtable}
\newpage
\subsection{Testy Niefunkcjonalne}
\begin{longtable}{|p{4cm}|p{9cm}|p{2cm}|}
\hline
\textbf{Rodzaj testu} & \textbf{Opis i Wynik} & \textbf{Status} \\
\hline
Test Bezpieczeństwa (Bandit) & Skanowanie statyczne kodu (SAST). Wynik: Brak zagrożeń (Severity: High/Medium). Naprawiono problem ze ścieżką /tmp. & Zaliczony \\
\hline
Test Wydajności (Locust) & Symulacja obciążenia 100 użytkowników. Średni czas odpowiedzi dla /api/analysis < 15ms. Obsługa błędów zewnętrznych API nie wpływa na stabilność serwera. & Zaliczony \\
\hline
\end{longtable}

\section{Wnioski}
Zrealizowano wszystkie zaplanowane zadania testowe dla modułu Video-Sent API. System wykazuje stabilność, poprawnie obsługuje błędy oraz scenariusze współbieżnego dostępu (duplikaty, konflikty). Kod jest zabezpieczony przed podstawowymi podatnościami, a wydajność samej aplikacji (narzut frameworka) jest na zadowalającym poziomie.
Rekomenduje się przejście do kolejnej fazy wdrożenia.

\end{document}
